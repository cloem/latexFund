\documentclass{article}
   \usepackage[utf8]{inputenc}
   \usepackage[ngerman]{babel}
   \usepackage{fullpage}
  \usepackage[unicode=true, 
   bookmarks=true,bookmarksnumbered=false,bookmarksopen=false,
   breaklinks=false,pdfborder={0 0 0},backref=false,colorlinks=false]
   {hyperref}
  \hypersetup{pdftitle={Wertung der Reformen Solons},
   pdfauthor={Charly Walther},
   pdftex}
   \title{Wertung der Reformen Solons}
   \author{Charly Walther}
   \begin{document}
   \maketitle

                      
Die Polis Athen befand sich im 7. und 6. Jahrhundert vor Christus in einer schweren sozialen und wirtschaftlichen Krise. Ein durch Überbevölkerung und Erbteilung ausgelöster Mangel an Land für die Bauern, sowie eine durch die aufkommende Münzprägung ermöglichte Vergabe von Krediten, führten viele Kleinbauern in die Schuldknechtschaft. Während der aufblühende Handel der Oberschicht zu Gute kam, da sie sich Investitionen leisten konnten, war die Existenz von einfachen Handwerkern bedroht durch die billige Arbeitskraft der Sklaven. Zusammen führten diese Faktoren zu einer immer größer werdenden Kluft zwischen Arm und Reich. Die Entmündigung großer Teile der Bürger der Polis hatte somit eine Destabilisierung der politischen Ordnung zufolge und bedeute eine Gefahr für das Fortbestehen der Polis.  \\
Um dem entgegen zu wirken und etwaige Gefahren wie einen Bürgerkrieg abzuwenden, wurde im Jahr 594 vor Christus der Gelehrte Solon, welcher Grieche war und aus dem Adel stammte, zum Gesetzgeber ernannt und beauftragt Reformen zu entwickeln, welche den Zusammenbruch der Polis Athen verhindern sollten. Solons Ziel war es, zur guten Ordnung, der \textit{eunomia}, zurückzukehren. Als Ursachen der Krise sah er die Schuldknechtschaft, die Verbrechen der Bürger und die Maßlosigkeit der Reichen. In der von ihm verfassten Schrift "`Elegie"' nennt ihr diese Probleme das "`verbogene Recht, das vermessene Wollen"', welche es zu beseitigen galt. \\
In seinen Reformen schaffte Solon zunächst die Schuldknechtschaft ab und verfügte einen Erlass der privaten und öffentlichen Schulden (genannt \textit{seisáchtheia} - Lastenabschüttelung). Um handwerkliche Tätigkeiten zu fördern, erließ er ein Gesetz, das besagte, lässt ein Vater seinen Sohn kein Handwerk erlernen, muss der Sohn ihn nicht unterhalten. Mit dem Testierrecht erlaubte er es kinderlosen ihr Erbe zu vermachen, wem sie wollten.  \\
Neben den wirtschaftspolitischen Maßnahmen, von denen vor allem die Abschaffung der Schuldknechtschaft sowie die Lastenabschüttelung bedeutend waren, verfügte Solon weitreichende Veränderungen in der sozialen Gliederung des Demos. Er teilte die Bevölkerung in vier Klassen nach ihrem Ertrag und wies den einzelnen Klassen konkrete Rechte und Pflichten zu, sowohl für das zivile Leben, als auch für militärische Aufgaben. Zur obersten Klasse der Pentakosiomedimnoi gehörte, wer 500 Maß\footnote{gemessen in Scheffel, ein Scheffel entspricht circa 50 Litern} an Ernte einbringen konnte. Im Kriegsfall hatten sie die Kosten für die Flotte zu decken. Die zweite Klasse der Hippeis, mit einem Mindestertrag von 300 Maß, war im Krieg als Reiter tätig. Die dritte Klasse der Zeugiten musste mindestens 200 Maß Ernte einbringen und sie wurden im Krieg als Hoplithen eingesetzt. Alle die nicht mindestens einen Ertrag von 200 Maß hatten, gehörten der Klasse der Theten an, im Krieg waren sie als Leichtbewaffnete oder Ruderer tätig. Zu den politischen Ämtern hatten nur die oberen Schichten Zugang, die Theten nahmen nur an der Volksversammlung (\textit{ekklesia}) und an den Volksgerichten (\textit{dikasterion}) teil. Somit hatte Solon grundlegend eine timokratische Verfassung geschaffen. Zur sozialen Gliederung und territorialen Verwaltung schaffte Solon die Einteilung der Bürger in eine von vier Phylen, Stämme mit einem Stammesführer (Phylenkönig) an der Spitze. In den Phylen waren die Klassen gemischt, eingeteilt wurde nach Wohnlage. Desweiteren wird Solon die Erschaffung eines "`Rats der Vierhundert"' nachgesagt, bestehend aus je einhundert Bürgern jeder Phyle, wobei die Theten auch hier kein Mitspracherecht gehabt haben sollen. Die Existenz des Rats der Vierhundert ist allerdings historisch nur durch wenige Quellen belegt.  \\
Solons Reformen haben es geschafft, ein Größerwerden der Kluft zwischen Arm und Reich zu verhindern, eine Annäherung zu erreichen, und diese auch zu sichern, da er nicht nur die Schulden hat tilgen lassen, sondern die Schuldknechtschaft generell abgeschafft hat. Der befreiten Unterschicht hat er weiterhin geholfen durch die Stärkung des Handwerks. Daraus ergab sich ein Proletariat, welches unabhängiger von der aristokratischen Oberschicht war. Mehr Freiheit wurde den Bürgern auch durch das Testierrecht gegeben, dank dem sie freier über ihren Nachlass entscheiden konnten. Dank der Reformen hatte nun auch die Unterschicht ein aktives Wahlrecht und Zugang zur Volksversammlung, womit sie zumindest theoretisch weit mehr Einflussnahme auf politische Entscheidungen ausüben konnten, als unter Drakon. Am stärksten war jedoch sicherlich die Verschiebung der Macht von einer Herrschaft der Aristokraten zu einer mehr demokratisch geprägten Machtverteilung in den Volksgerichten, da dort nun auch die vierte Klasse der Theten, welche den größten Teil der Bevölkerung ausmachte, an den Abstimmungen teilnehmen durfte. Betrachtet man nun diese wirtschaftlichen Maßnahmen und die Veränderungen der Zugänglichkeit zu den politischen Institutionen, stellt man fest, dass Solon damit seine Hauptanliegen erfolgreich angegangen ist. Er hat verhindert, dass die Oberschicht sich weiter maßlos durch die Schuldknechtschaft bereichern kann, und verhindert des Weiteren, dass die Oberschicht sich allein ihre Gesetze schafft und allein über sie richtet, indem er die Unterschicht mit in die Volksgerichte geholt hat.  \\
In der Praxis mangelte es aber an einigen wichtigen Punkten, sodass man auf keinen Fall sagen kann, Solon hätte ganz im Sinne der armen Bevölkerung gehandelt, was um es nochmal zu betonen, auch gar nicht sein Ziel war. So war es dem Großteil der Unterschicht, vor allem der Bauern, überhaupt nicht möglich, ihre Rechte in der Volksversammlung oder dem Volksgericht wahrzunehmen. Da es keinerlei Diäten oder ähnliches gab, war dies nur denjenigen möglich, die es sich leisten konnten ihre Arbeit niederzulegen für die Zeit, welche die Teilnahme an der Volksversammlung oder am Volksgericht in Anspruch genommen haben. Es ist anzunehmen, dass die wenigsten Bauern es sich leisten konnten ihr Feld für einen Tag zu verlassen. Auch an dem Ursprung der Krise, der Verteilung des Landes ändert Solons Reform nichts. Darin dass Solon selbst die Landverteilung gar nicht als Ursache der Krise sah und er auch nicht das Ziel hatte, diese zu korrigieren, bestätigt sich, dass Solon zu keinem Zeitpunkt der große Heilsbringer für das arme Volk sein wollte, und nie gegen die Interessen der Aristokratie arbeiten wollte. Solons Ziel war nur, den Zusammenbruch der Polis zu verhindern. Somit erübrigt sich auch die Frage, ob er mehr hätte erreichen können, da dies gar nicht sein Ansinnen war. Das passive Wahlrecht blieb bei den Vermögenden, das Archontat war aristokratisch, das Areopag oligarchisch, wie Aristoteles richtig feststellte. Dass in Solons Reformen die Grundzüge einer demokratischen Idee zu erkennen sind, kann man nicht abstreiten, auch dass die Unterschicht sozial gestärkt aus den Reformen hervorging nicht, aber von einer Demokratie kann noch lange nicht gesprochen werden. Hierzu fehlt die echte Einflussnahme in den politischen Ämtern und überhaupt erstmal die praktische Möglichkeit aller an der Volksversammlung und den Volksgerichten teilzunehmen.  \\
Die Reformen Solons kann man als einen Kompromiss sehen zwischen einem Ausgleich der Diskrepanz zwischen Arm und Reich und der Wahrung der Vormachtstellung des Adels. Solon hat erfolgreich dem verbogenen Recht zur Rückkehr zur guten Ordnung verholfen, hatte dabei aber nicht den Anspruch eine demokratische Verfassung zu erschaffen. 
  \end{document}
                   
