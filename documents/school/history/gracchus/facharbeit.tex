\documentclass[
    12pt,
    smallheadings,
    ]{scrreprt}

\usepackage[utf8]{inputenc}
\usepackage{ngerman}
\usepackage{geometry}

\usepackage{hyphenat}

\geometry{a4paper, left=3.5cm, right=2.5cm, top=2.5cm, bottom=1.5cm}


\usepackage{scrpage2}
\clearscrheadfoot
%\chead[zentriert - oben - kapitelanfang]{zentriert - oben - textseiten}
\chead[-~\pagemark~-]{-~\pagemark~-}
%\ohead[rechts - oben - kapitelanfang]{rechts - oben - textseiten}
\pagestyle{scrheadings}

\renewcommand{\baselinestretch}{1.50}\normalsize

\usepackage[unicode=true, 
 bookmarks=true,bookmarksnumbered=false,bookmarksopen=false,
 breaklinks=false,pdfborder={0 0 0},backref=false,colorlinks=false]
 {hyperref}
\hypersetup{pdftitle={Die Brüder Gracchus},
 pdfauthor={Thomas Wienecke},
 pdftex}


% Römische Zahlzeichen
\newcommand{\RM}[1]{\MakeUppercase{\romannumeral #1}}

%%%%%%%%%%%%%%%%%%%%%%%%%%%%%%%%%%%%%%%%%%%%%%%%%%%%
% --------------- start of content --------------- %
%%%%%%%%%%%%%%%%%%%%%%%%%%%%%%%%%%%%%%%%%%%%%%%%%%%%


\begin{document}


\title{Die Brüder Gracchus \\
Retter des Staates oder Anarchisten?}
\author{Thomas Wienecke}
\date{\today}
\maketitle

% chapterstyle auf empty -> keine Seitenzahl
\renewcommand*{\chapterpagestyle}{empty}
\tableofcontents
\clearpage

% chapterstyle wieder auf plain
\renewcommand*{\chapterpagestyle}{plain}

\chapter{Aufgabe}
Die Brüder Gracchus: Retter des Staates oder Anarchisten? \\
Darstellung der Situtation der römischen Kleinbauern, Interessen des Patriziats; Untersuchung und Bewertung der Methoden der Gracchen v. a. im Hinblick auf die römische Verfassung

\chapter{Einleitung}

Die Brüder Gracchus, die von ca. 162 v.Chr.\footnote{Im Folgendem verzichte ich auf die Verwendung von 'v.Chr', da sich alle Jahreszahlen auf die Zeit vor Christus beziehen}
bis 122 lebten, versuchten die römische \textit{res publica} zu reformieren.
Dabei gingen sie mit dem Ackergesetz besonders auf das Agrarproblem ein, welches durch die Punischen Kriege und die damit verbundene römische Expansion enstanden war.
Am Ende scheiterten jedoch beide am Widerstand ihrer Gegner und kamen aufgrund ihrer politischen Aktivitäten ums Leben.
Dem Wirken der Gracchen folgten dann der römische Bürgerkrieg und der Untergang der Republik bzw. der Beginn der römischen Kaiserzeit.

Ob und inwiefern man die Gracchen als Retter des Staates oder als Anarchisten bezeichnen kann, möchte ich im Folgendem untersuchen.




\chapter{Die römische Gesellschaft zur Zeit der Gracchen}
    \section{Die Agrarkrise und ihre Ursachen}
Nach den Punischen Kriegen kam es in Rom zu einer Agrarkrise, die mehrere Ursachen hatte, die sich vorallem aus den Punischen Kriegen erschließen lassen.
Die Agrarkrise führte dazu, dass die Großgrundbesitzer immer mehr Acker kauften oder auch einfach besetzten und somit viele Kleinbauern um ihr Land brachten.
Diese Kleinbauern, die daraufhin besitzlos waren, teilten sich in das städtische und das ländliche Proletariat auf.
Das ländliche Proletariat arbeitete zusammen mit Sklaven auf den Äckern der Großgrundbesitzer, während das städtische Proletariat hauptsächlich nach Rom zog.

        \subsection{Das Ende der Kolonisation}
In den drei Punischen Kriegen von 264 bis 146 hatten sich die Römer die Herrschaft über den westlichen Teil des Mittelmeerraums erkämpft.
Dabei hatten sie, nachdem sie fast selbst vernichtend geschlagen worden waren, die Karthager besiegt.
Sowohl die drei großen Inseln des westlichen Mittelmeeres -- Sardinien, Corsica und Sizilien -- als auch der nördliche Teil Afrikas, standen nun unter der Kontrolle Roms.

Da Philipp \RM{5}. von Makedonien sich im Zweiten Punischen Krieg mit dem karthagischen Heerführer Hannibal verbunden hatte und auch die Aitolier ein Bündnis mit Rom gegen Makedonien frühzeitig aufgaben, hatte Rom allen Grund auch im Osten, in Griechenland, durchzugreifen.
Somit schwächte es in den Makedonisch-Römischen Kriegen ein weiteres Gefahrenpotenzial ab und hatte nun Einfluss über den gesamten Mittelmeerraum erlangt.

Die Beseitigung der größten außenpolitischen Gefahren zog allerdings auch innenpolitische Konsequenzen nach sich.
Die römische Expansion war weitestgehend zum Stillstand gekommen und mit ihr die Kolonisation.
Kolonien wurden hauptsächlich in neu erorberten Gebieten gegründet, um diese unter Kontrolle zu halten.
Nebenbei sorgten sie allerdings auch für die Versorgung der Veteranen mit Land, denn vorallem in die \textit{coloniae civium romanorum} wurden römische Staatsbürger geschickt, welchen per Los ein Stück Land gegeben wurde.


        \subsection{Die Latifundienwirtschaft}
\label{Latifundien}
Nach den Punischen Kriegen breitete sich die Latifundienwirtschaft bei den Römern aus.
Latifundien waren große Äcker, auf denen ein reicher Gutsherr Sklaven oder freie Bürger anstellte.
Meistens handelte es sich dort um Oliven- oder Weinanbau, da dieser rentabler war, als der Anbau von Getreide.
Zu dieser Zeit waren Teile der Nobilität aufgrund der \textit{lex claudia de nave senatorum}\footnote{Gesetz aus dem Jahre 218 vom Volkstribun Claudius, welches den Angehörigen des Senats den Besitz von Schiffen verbot, die mehr als 300 Amphoren (römische Maßeinheit; entspricht ca. 26 Litern) transportieren konnten} so sehr bestrebt, ihren Besitz zu vergrößern, dass sie auch nicht vor der gewaltsamen Übernahme von Kleinbauernstellen zurück schreckten.

Durch die Punischen Kriege, die sich über viele Jahre hinweg zogen, blieben viele Bauern eine lange Zeit im Krieg und konnten sich nicht selbst um ihr Land kümmern.
Bauern von kleinen und mittleren Gutshöfen hatten jedoch nur wenige Angehörige der Familie, die zurück blieben, um die Felder zu bewirtschaften.
Wenn die Familie zu groß für den Acker war und es schwer war alle Angehörigen zu versorgen, konnte das sogar von Vorteil sein. 
Wurde nun ein Sohn zum Militärdienst eingezogen, so musste das Militär sich um seine Ernährung kümmern und die Familie hatte eine Person weniger zu alimentieren.
Nachteilig war es jedoch, wenn Vater und Söhne für mehrere Jahre in den Krieg mussten und fast nur noch weibliche Familienmitglieder zurückblieben und mit der Bewirtschaftung der Felder überfordert waren.
Auch die Kriegsbeute reichte dann für die ärmere Bevölkerung meist nicht aus, um die Missstände wett zu machen.

Die Besitzer von großen Gutshöfen hatten damit nicht allzu große Schwierigkeiten.
Sie hatten die Möglichkeit freie Bauern oder Sklaven anzustellen, weshalb ihr Land in ihrer Abwesenheit nicht verkümmern musste.
Deshalb konnten sie mit ihrem Kapital Druck auf die Kleinbauern, die gerade aus dem Krieg wieder kamen, ausüben oder in ihrer Abwesenheit einfach ihren Acker besetzen.


Im Extremfall kann man sich also einen Bauern vorstellen, der gerade nach Jahren, die er im Krieg in Afrika verbracht hat, wieder zu seinem Gut zurückkehrt, welches er recht verwahrlost wiederfindet.
Während er bemüht ist den Schaden, den seine Abwesenheit angerichtet hat, wieder gutzumachen, kommt der reiche Gutsbesitzer, der den schon riesigen Acker von nebenan besitzt und macht ihm das Angebot sein Land zu kaufen.
Vielleicht lehnt der Kleinbauer die ersten Angebote noch ab, aber nach einer Weile wird ihm viel mehr geboten, als sein Acker überhaupt wert ist und er verkauft ihn womöglich.
Jetzt, da er kein Land mehr besitzt, muss er sich entscheiden, ob er auf dem Feld eines reichen Gutsbesitzers arbeiten möchte oder ob er nach Rom geht, um dort von \textit{panem et circenses}\footnote{"`Brot und Wagenrennen"' oder öfter übersetzt als "`Brot und Spiele"'} zu leben.
So oder so ähnlich kam viel Land der Bauern mit kleineren und mittleren Ackerflächen in den Besitz von Großgrundbesitzern der \textit{nobiles}.






%        \subsection{\textit{lex claudia de nave senatorum}}
%''Nur so kann man begreifen, daß im Jahre 218 v.Chr. ein Gesetz durchging, das für die künftige innere Entwicklung Roms schlechthin fundamentale Bedeutung gewinnen sollte. Es verbot den Senatoren den Besitz von Schiffsraum, der über ein mittleres Maß hinausging, genug, um ihnen den Transport der eigenen Landerzeugnisse zu gewährleisten (lex Claudia). Der Zweck war also, ihnen den Handel als selbständigen Erwerbszweig zu verschließen. Es bestanden demnach Tendenzen zu einer Kommerzialisierung des Senatorenstandes.''\cite{heuss}\\
%Die \textit{lex claudia de nave senatorum} aus dem Jahre~218 verbot den Angehörigen des Senats und deren Kindern Schiffe zu bauen, die mehr als 300~Amphoren\footnote{römische Maßeinheit; entspricht ca. 26 Litern} transportieren konnten.
%Das Gesetz sollte die Konkurrenz zwischen importierten Waren und den Produkten aus Italien verhindern und gleichzeitig dafür sorgen, dass das Geld der reichen Senatoren nur in die eigene Landwirtschaft investiert würde.

\chapter{Die Methoden der Gracchen}
    \section{Tiberius Sempronius Gracchus}
        \subsection{Leben bis zur Reform}
Tiberius Sempronius Gracchus kam um das Jahr 162 zur Welt.
Nur drei von 12 Kindern der Familie wuchsen zu Erwachsenen heran, von denen Tiberius der älteste war.\footnote{vgl. Grimal, Pierre (Hrsg.), Der Aufbau des römischen Reiches -- Die Mittelmeerwelt im Altertum~\RM{3}., S.~110}
Durch seine Eltern war Tiberius Sempronius Gracchus mit den drei adeligsten Geschlechtern Roms verwandt.
Seine Mutter Cornelia war die Tochter der Aemilia, aus der Familie Aemilii Paulli und des Publius Cornelius Scipio Africanus, welcher im Zweiten Punischen Krieg den Sieg über Hannibal herbeiführte.\footnote{Baker, Simon, Rom - Aufstieg und Untergang einer Weltmacht, S.~46}
Durch die freundschaftlichen Kontakte seiner Mutter zu Diophanes von Mytilene und Blossius von Cumae wurde Tiberius von griechischen Gelehrten unterrichtet.\footnote{Bengtson, Hermann, Römische Geschichte - Republik und Kaiserzeit bis 284~n.Chr., S.~123}

Das erste politische Amt übernahm Tiberius als \textit{quaestor} für das Jahr 137, dem unterstem Magistrat der römischen Verfassung, welches als Hilfsmagistrat des Konsuls und zur Verwaltung der Staatsfinanzen diente.
In diesem Jahr wählte Gaius Hostilius Mancinus Gracchus aus, um ihn als Finanzoffizier auf seiner Expedition nach Spanien zu begleiten.
Auf dem Weg dorthin durchquerte Tiberius Etrurien, die italische Landschaft im Norden von Rom, und sah dort die Konsequenzen, welche die römische Expansion mit sich gebracht hatte.
So begnete er zum einen landwirtschaftlichen Großbetrieben mit angestellten, ausländischen Sklaven oder freien Bürgern des ländlichen Proletariats.
Im Gegensatz dazu traf er Kleinbauern, "`die durch den Tod ihrer männlichen Familienangehörigen gezwungen waren, ihren Grund und Boden zu verlassen"'\footnote{Ebd., S.~81}
oder die unter der Konkurrenz der importierten Getreidewaren und dem Druck der Großgrundbesitzer, zusammenbrachen.\footnote{vgl. S.~\pageref{Latifundien}}

\label{mancinus}
Im folgendem Krieg in Spanien kam das Heer von Mancinus in den Hinterhalt und Tiberius sollte als römischer Vertreter einen Vertrag mit den Numantinern aushandeln.
Dieser Vertrag rettete dem Heer, das hauptsächlich aus Kleinbauern -- Tiberius' späterem Klientel -- bestand, das Leben.
In Rom akzeptierte der Senat diesen Vertrag jedoch nicht und Tiberius hatte Glück, dass er deswegen nicht, "`wie der Konsul Mancinus als Ersatzopfer~\lbrack ...\rbrack~an die Numantiner überstellt~\lbrack wurde\rbrack."'\footnote{Jehen, Martin, Die Römische Republik - Von der Gründung bis Caesar, S.~83}

Somit wurde diese Reise nach Spanien gleich in zweierlei Hinsicht zu einem Schlüs\-sel\-er\-eig\-nis für Tiberius.
Zum einen sah er den gravierenden Unterschied zwischen den überforderten Kleinbauern und den sich immer mehr ausbreitenden Großgrundbesitzern in Etrurien.
Zum anderen erkannte der Senat seinen Vertrag nicht an, obwohl er so vielen Römern das Leben rettete.
Dieses Verhalten des Senats traf auf Unverständnis beim jungen Tiberius und war wohl ein Grund dafür, dass er 4 Jahre später Politik nicht wie üblich vom Senat aus, sondern mit Hilfe der Volksversammlung gegen den Senat machte.
Sein unbeugsamer Wille die Gesetze zur Agrarreform durchzusetzen, kosteten ihm letztendlich das Leben.



        \subsection{Gesetzesentwurf zur Agrarreform}
Den meisten \textit{nobiles} war bewusst, dass das Militärwesen Roms auf den Schultern der Kleinbauern lag und dieses die Grundlage für die römische Expansion war.
Ohne Eigentum konnte ein Bauer nicht in das Heer aufgenommen werden und somit bedeutete jeder besitzlose Bürger einen Soldaten weniger.
Da die spanischen Kriege schätzungsweise 200 000 Mann das Leben gekostet hatten\footnote{vgl. Heuss, Alfred, Römische Geschichte, S.~144}
und man die Vorherrschaft über den Mittelmeerraum aufrecht erhalten wollte, musste also etwas gegen die Verarmung der Plebejer gemacht werden.

Deshalb gab es einige Versuche das Bauerntum und damit das ganze Milizwesen zu stärken.
Laelius, ein Freund des jüngeren Scipios, beantragte beispielsweise im Jahr 140 ein "`Gesetz zur Stärkung des Bauerntums auf Kosten der reichen Besitzer"'.\footnote{Ebd., S.~144}
 Dieses traf jedoch auf heftigen Widerstand seitens der Großgrundbesitzer und wurde somit nicht durchgesetzt.
\label{scaevola}Ein paar Jahre später versammelten sich um Appius Claudius Pulcher\footnote{Konsul von 143} unter anderem zwei Juristen, Publius Mucius Scaevola und Publius Licinius Crassus Dives Mucianus, welche es sich ebenfalls zur Aufgabe machten, das Agrarwesen zu reformieren.
Sie arbeiteten ein Gesetz aus, welches das Staatsland -- den \textit{ager publicus}\footnote{bezeichnet den Teil des römischen Landes, den die Römer ihren Gegner nach einem Sieg für gewöhnlich abnahmen; römische Staatsbürger konnten dieses Land gegen eine kleine Miete okkupieren (\textit{ager occupatorius})}
-- an besitzlose Bürger verteilen sollte.
Zunächst sah es der Gesetzesentwurf vor, ein altes -- vermutlich in Vergessenheit geratenes oder einfach unbeachtetes -- Gesetz wieder in Kraft treten zu lassen und zu erneuern.
Es legte das Maximum an Staatsland, das ein Bürger okkupieren durfte, auf 500 Morgen fest.
Die Erneuerung bedeutete zum einen, dass die Grenze für die ersten beiden Söhne jeweils um 250 Morgen angehoben wurde, sodass man mit zwei Söhnen 1000 Morgen vom \textit{ager publicus} besetzen konnte.\footnote{Ebd., S.~146}
Zum anderen sollte das Land, welches bis zum Zeitpunkt des Inkrafttretens des Gesetzes bereits okkupiert war und die Höchstgrenze nicht überschritt, "`in freies Eigentum, das gegen staatliche Rückforderungsansprüche geschützt war, überführt werden."'\footnote{Bringmann, Klaus, Krise und Ende der römsichen Republik (133-42~v.Chr), S.~45}
Der dadurch frei werdende Teil des \textit{ager publicus} sollte dann in Parzellen zu je 30 Morgen an die Armen verteilt werden.
Damit das Gesetz nicht wieder unbeachtet in Vergessenheit geraten konnte, sollte sich eine aus drei Männern bestehende Ackerkommission um die Durchsetzung kümmern und würde dafür mit richterlicher Gewalt ausgestattet werden.
Der Entwurf sah vor, dass die Männer wie Magistrate vom Volk gewählt werden und den Titel \textit{triumviri agris iudicandis adsignandis} erhalten würden.\footnote{vgl. Appianos von Alexandria, Bellum Civile, 1,37}


%Normalerweise hätte ein Volkstribun diesen Gesetzesentwurf als erstes vor den Senat gebracht, damit dieser darüber beraten konnte.
%Ti. Gracchus tat dies jedoch nicht, sondern stellte den Entwurf sofort der Volksversammlung vor.
%Tiberius wusste wohl, dass er sich vom Senat nicht viel Zustimmung erhoffen brauchte.
%Denn die Mitglieder des Senats waren vor allem diejenigen, welche mehr als die Höchstgrenze von 1000 Morgen am \textit{ager publicus} besaßen und somit durch dieses Gesetz eine gewisse Menge an Land verlieren würden.



        \subsection{Abstimmung und Widerstand}
Obwohl Scaevola einer der beiden Konsuln im Jahre 133 war, sollte Tiberius, welcher begeistert von der Reformidee war, das Gesetz mit Hilfe des Volkstribunats durchsetzen.\footnote{vgl. Heuss, Alfred, Römische Geschichte, S.~144}
Anstatt, wie damals üblich, zuerst den Senat mit dem Gesetz zu konfrontieren und diesen intern darüber entscheiden zu lassen, ging er auf direktem Wege vor die Volksversammlung, damit diese über das Gesetz abstimmen konnte.
Ohne die Zustimmung der Senatsmehrheit, konnte man für gewöhnlich kein neues Gesetz mit Hilfe der Volksversammlung abschließen, obwohl der Senat laut der römischen Verfassung nur eine beratende Funktion hatte.
Indem sich die Senatsmitglieder einen der 10 Volkstribunen -- sei es durch Bestechung, Überredungskünste oder Ähnlichem -- auf ihre Seite holten, konnten sie mit Hilfe seines Vetorechts jede Abstimmung kippen.
Wer noch eine weitere politische Karriere anstrebte, war also besser damit beraten, den Senat zu konsultieren und seine Entscheidung zu akzeptieren, um seinen Unterstützerkreis nicht zu minimieren.

Tiberius hielt sich also nicht an diese Reihenfolge und machte ohne eine Senatsentscheidung von seinem Recht Gebrauch, die Volksversammlung einzuberufen.
Bevor es zur Abstimmung kam, hielt er eine Rede, in welcher er behauptete, dass die Bauern, die für Italien kämpfen und sterben würden nichtmal wie die wilden Tiere eine Unterkunft hätten und nur für den Reichtum und Wohlstand Anderer kämpfen würden.\footnote{vgl. Plutarch, Tiberius Gracchus, 9,5-6}
Die Volksversammlung war infolgedessen auf seiner Seite und wäre es zur Abstimmung gekommen, so wäre das Gesetz mit Sicherheit angenommen worden.
Doch "`die Angst vor der Herrschaft der Masse, dann das Mißtrauen gegen die Aktivität des Volkstribunen und nicht zuletzt die begründete Furcht vor größeren Land- und Vermögensverlusten"'\footnote{vgl. Bengston, Hermann, Römische Geschichte - Republik und Kaiserzeit bis 284~n.Chr, S.~130}
sorgten für eine heftige Opposition vieler \textit{nobiles}.
Sie fanden ihren Vertreter in Tiberius' Kollegen Marcus Octavius, der sein Veto einlegte und das Gesetzesvorhaben somit blockierte.
Tiberius hatte anscheinend nicht damit gerechnet, dass ein Vertreter des Volkes sein Veto gegen dieses Gesetz einlegen könnte.
Doch da Octavius trotz aller Proteste auf seiner Entscheidung beharrte, blieb dem jungen Gracchus nichts anderes übrig, als diese Niederlage vorerst hinzunehmen.

        \subsection{Verfassungsbrüche und Untergang}
Er hätte bis zur nächsten Volkstribunenwahl warten und einen seiner Nachfolger damit beauftragen können, das Gesetz zu verabschieden, in der Hoffnung, dass niemand wie Octavius Gebrauch von seinem Vetorecht machen würde.
Das würde allerdings bedeuten, dass Tiberius ein zweites Mal -- nach der Mancinus-Affäre\footnote{vgl. S.~\pageref{mancinus}}~-- sich politisch nicht durchsetzen könnte.
Stattdessen verwendete er zunächst "`sein eigenes Interzessionsrecht zu Repressalien gegen die Staatsverwaltung"'\footnote{Heuss, Alfred, Römische Geschichte, S.~145}, was jedoch nur zur Verhärtung der Gegensätze führte.

Da alles nichts half, entschied er sich für einen äußerst revolutionären Weg und lies Marcus Octavius durch die Volksversammlung aus seinem Amt absetzen.
Damit missachtete er einen der wichtigsten Grundsätze der römischen Verfassung, "`die Unverletzlichkeit des Volkstribunen"'.\footnote{Ebd., S.~145}
\label{octavius}Er argumentierte damit, dass ein Volkstribun seine Unverletzlichkeit nicht behalten dürfte, sobald er sich gegen das Volk wendete, da es das Volk war, welches ihm seine Macht und Unverletzlichkeit verliehen hatte.\footnote{vgl. Plutarch, Tiberius Gracchus, 15,2-9}
Die Volksversammlung war jedenfalls auf Tiberius' Seite und ersetzte Marcus Octavius durch einen neuen Vertreter, der den Agrarreformen positiv gestimmt war.
Dadurch stand dem Gesetz nichts mehr im Wege und es wurde angenommen.

Mit diesem Verfassungsbruch nahm er den Senatoren ihr Werkzeug aus der Hand und entfesselte eine Revolution, die Rom die nächsten vierzig Jahre prägen sollte.
Das Volkstribunat wurde "`wieder eine revolutionäre Instanz und mit dem Anspruch ausgestattet, als selbstständige politische Mitte zu gelten und im Verein mit dem Volk gegebenenfalls zu regieren."'\footnote{Heuss, Alfred, Römische Geschichte, S.~145}
Das Gesetz trat also in Kraft und als erstes \textit{triumvirat} wurden Tiberius Gracchus selbst, sein Schwiegervater Appius Claudius Pulcher und sein jüngerer Bruder Gaius Gracchus gewählt.
Da die Umsetzung der \textit{lex Sempronia} viel kostete und der Senat verständlicherweise nicht bereit war, dieses beizusteuern, machte Gracchus es sich zunutze, dass der kürzlich verstorbene König Attalos \RM{3}. von Pergamon sein Vermögen testamentarisch den Römern übertragen hatte.
Erneut benutze er die Volksversammlung, um abzustimmen, dass das Geld für das Ackergesetz verwendet werden sollte.
Damit beging er einen weiteren Verfassungsbruch, da er den Senat wieder umging, der für die finanziellen Angelegenheiten in Rom zuständig war.\footnote{vgl. Bringmann, Klaus, Krise und Ende der römischen Republik (133-42~v.Chr), S.~46}

Spätestens jetzt hatte er den Großteil des Senats gegen sich aufgebracht und musste nach Ablauf seiner Amtszeit -- dem Ende seiner Immunität -- damit rechnen, dass seine Gegner ihm wegen seiner Verfassungsbrüche den Prozess machen würden.
Tiberius Sempronius Gracchus strebte deshalb noch während seiner ersten Amtszeit eine Zweite an.
Es war zu der Zeit schon unüblich genug, zweimal zum Volkstribun gewählt zu werden, aber dies gleich in zwei aufeinander folgenden Jahren zu tun, war im Prinzip ein weiterer Verfassungsbruch.
Durch dieses Vorgehen hatte Tiberius die Stimmung im Senat wohl endgültig zum Kochen gebracht und dieser beauftragte den Konsul Mucius Scaevola gegen die Wiederwahl einzuschreiten.
Scaevola war jedoch ein Freund der Reform\footnote{vgl. S.~\pageref{scaevola}}
und weigerte sich diesem Befehl Folge zu leisten, indem er auf das Prinzip der Unverletzlichkeit des Volkstribunen bestand.
Seine Verweigerung konnte Tiberius allerdings nicht retten, denn viele Senatsmitglieder stürmten unter Führung von Scipio Nasica in die Volksversammlung und erschlugen ihn und viele seiner Anhänger.\footnote{vgl. Jehne, Martin, Die Römische Republik -- Von der Gründung bis Caesar, S.~84}


%Um die Kleinbauern, die eine Parzelle erhalten würden mit dem nötigen Inventar auszustatten, welches für die Bewirtung eines Ackers nötig war, wollte man den Staatsschatz verwenden, den Attalos \RM{3}., König von Pergamon, testamentarisch den Römern vermacht hatte.

        \subsection{Von Tiberius zu Gaius Gracchus}
Die breite Masse der römischen Bevölkerung sah in Tiberius einen "`unschuldig hingeschlachteten Märtyrer\lbrack\ldots\rbrack"' und war nicht geneigt seine "`Vernichtung~\lbrack\ldots\rbrack~moralisch anzuerkennen."'\footnote{Heuss, Theodor, Römische Geschichte, S.~147}
Der Senat betonte deshalb, dass nicht die Agrarreform als inhaltlicher Sachverhalt, sondern das Vorgehen von Tiberius und seinen Anhängern das Problem darstellten.
Die Regierenden trauten sich daher zunächst auch nicht, dass Agrargesetz anzutasten oder gar abzuschaffen.
Als aber im Jahre 129 die Ackerkommission in Konflikt mit nichtrömischen Italikern kam -- ihnen wurde Land abgenommen, aber keines an sie verteilt --, entzog der jüngere Scipio\footnote{Publius Cornelius Scipio Aemilianus Africanus, Sohn des Publius Cornelius Scipio Africanus, welcher im Zweiten Punischen Krieg den Sieg über Hannibal errung; zur besseren Unterscheidung \textit{Scipio, der Jüngere} genannt}
ihr kurzerhand die richterliche Befugnis, woraufhin diese faktisch handlungsunfähig wurde.
Die \textit{lex Sempronia} war alles andere als erfolgreich, denn nur etwa 3000 neue Bauernstellen waren durch sie bis zu diesem Jahr entstanden.\footnote{vgl. Bringmann, Klaus, Krise und Ende der römischen Republik (133-42~v.Chr.), S.~47}

Doch auch wenn es nach einem Sieg der Senatoren aussah, war Gaius Gracchus gemäß der römischen Familiensolidarität, "`geradezu verpflichtet \lbrack\ldots\rbrack, seinen Bruder zu rehabilitieren"'\footnote{Jehne, Martin, Die Römische Republik - Von der Gründung bis Caesar, S.~86}
und rüstete sich gemeinsam mit seinen Freunden zu einem erneuten Versuch, die Reform erfolgreicher durchzusetzen.

\label{flaccus}Die erste Vorbereitung traf er gemeinsam mit seinem Freund M. Fulvius Flaccus, \textit{triumvir agris iudicandis} seit 130 und Konsul für das Jahr 125, indem sie ein Gesetz verabschieden liesen, welches die zweimalige Wahl zum Volkstribun hintereinander legalisierte.
Nachdem Gaius 126 für zwei Jahre \textit{quaestor} in Sardinien war und sein Reformprogramm ausgearbeitet hatte, lies er sich zum Volkstribun für das Jahr 123 wählen.


    \section{Gaius Sempronius Gracchus}
        \subsection{Reformprogramm und erstes Tribunatsjahr}
Gaius Sempronius Gracchus hatte aus dem Schicksal seines älteren Bruders gelernt.
Er war der Meinung, dass Tiberius nur deshalb gescheitert war, weil er sich unüberlegt in den politischen Kampf begeben hatte und mit aller Gewalt das Ackergesetz durchbringen wollte.
Deshalb wollte er umsichtiger ans Werk gehen und sein Vorgehen besser voraus planen.
Er versuchte mit seinem rhetorischen Talent, dass sogar Jahre später vom politisch gegensätzlich orientiertem Cicero bewundert wurde\footnote{vgl. Grimal, Pierre (Hrsg.), Der Aufbau des römischen Reiches -- Die Mittelmeerwelt im Altertum~\RM{3}., S.~116~f.},
und seinem umfassendem Reformprogramm eine große Schar von Unterstützern aufzubauen.

Mit seinen ersten beiden Gesetzen wollte er aber zunächst die Ehre seines Bruders wieder herstellen.
Der Verfassungsbruch, welcher am meisten für Tiberius' Ende gesorgt hatte, war die Absetzung des Volkstribun Marcus Octavius.
Durch sein erstes Gesetz, der \textit{lex de abactis}, legalisierte er nicht nur diesen Entzug des Volkstribunats, sondern verwehrte dem Abgesetztem sogar die weitere politische Karriere.
Weiterhin erneuerte er mit der \textit{lex Sempronia de provocatione} das Provokationsrecht, mit dem er rückwirkend die außerordentlichen Gerichtskommissionen des Senats ungesetzlich machte.
Dadurch zwang er den Konsul P. Popillius Laenas ins Exil zu gehen, der 132 ohne richterlichen Beschluss die Anhänger von Tiberius zum Tod verurteilte.

Es folgten Gesetze zu einer umfassenden Reform.
Die \textit{lex agraria} setzte das Agrargesetz von Tiberius wieder in Kraft und sorgte dafür, dass die Ackerkommission ihre judikative Befugnis zurück bekam.
Außerdem erneuerte er es so, dass die Parzellen, die verteilt wurden, 200 statt nur 30 Morgen umfassten.
Durch die \textit{lex militaris} durfte niemand vor dem 17. Lebensjahr zum Militärdienst eingezogen werden, der Staat musste für die Ausrüstung aufkommen und durfte dies nicht vom Sold abziehen.
Weiterhin legte Gaius mit der \textit{lex frumentaria} fest, dass Getreide zu einem festen Preis für die Bevölkerung zur Verfügung stand.
Da dieser Preis meistens unter dem Marktwert lag, musste der Staat für die Differenzen aufkommen.\footnote{vgl. Bengston, Hermann, Römische Geschichte - Republik und Kaiserzeit bis 284~n.Chr, S.~132}

Die nun folgenden Gesetze sollten den Senat schwächen und dafür sorgen, dass sich der gerade erst bildende Stand der Ritter Gaius bei seinem politischem Kampf unterstütze.
Zum Einen war das die \textit{lex Sempronia de provincia Asiae}, die den Rittern eine weitere Vermögensquelle eröffnete.
Sie führte in der Provinz Asien das gleiche Steuersystem ein, wie es auch schon in Sizilien vorhanden war.
Jeder hatte den Zehnten an Rom zu entrichten und die Besonderheit war, dass die Verpachtung der Eintreibung dieser Steuer nur an Ritter vergeben wurde.
Außerdem bekam der Ritterstand durch die \textit{lex iudiciaria} Zugang zum Gerichtswesen, denn neben 300 Senatoren sollten auch 300 Ritter als Geschworene tätig sein.
Die gravierendste Konsequenz, des zweitgenannten Gesetzes war wohl der Umstand, dass nun der Gerichtshof für die finanziellen Unregelmäßigkeiten der Provinzialverwaltung aus der Hand des Senats glitt.
Somit konnten sich die Provinzialverwalter nicht mehr auf das Prinzip "`eine Krähe hackt der anderen kein Auge aus"'\footnote{deutsches Sprichwort} verlassen, da nun Angehörige des Ritterstandes über sie richteten.\footnote{vgl. Heuss, Alfred, Römische Geschichte, S.~152}
Desweiteren wurde der Senat mit der \textit{lex de provinciis consularibus} gezwungen bereits vor der Wahl die künftige Provinz des Konsuln festzulegen, sodass nicht der gewählte Konsul um seine Provinz falschen konnte.


        \subsection{Zweites Tribunatsjahr und Untergang}
Da es inzwischen kein Verfassungsbruch mehr war, wurde Gaius Sempronius Gracchus ohne Probleme zum zweiten Mal, für das Jahr 122, zum Volkstribun gewählt.
Zunächst ging er auf die Italikerfrage ein und griff einen älteren Antrag von Flaccus auf, der den Italikern das römische Bürgerrecht verleihen sollte.
Weiterhin wollte er die Kolonisation vorantreiben und verabschiedete dazu die \textit{lex de coloniis deducendis}, wodurch zwei neue Kolonien in Capua und Tarent gegründet wurden.
Er wusste jedoch, dass der \textit{ager publicus} in Italien nicht genügend Land bot, um umfangreich zu kolonisieren und jedem Anwerter eine Parzelle zuzuteilen.
Außerdem wollte er es vermeiden, erneut mit den reichen Großgrundbesitzern in Mittelitalien in Konflikt zu geraten, die nach wie vor einen großen Teil des Staatslandes besetzten.
Deshalb kam es ihm gerade gelegen, dass sein Amtskollege, der Volkstribun Rubrius, ein Gesetz beantragte, welches die Aussendung einer Kolonie nach Karthago beinhaltete.
Dort sollten neben Römern ebenfalls gleichberechtigt die Italiker Land zugewiesen bekommen.
Gaius war unter Anderen ein Verantwortlicher für diese Kolonie, die \textit{Junonia} hieß, und musste sich nach Afrika begeben, um den Aufbau zu organisieren.

Diese Abwesenheit trug maßgeblich zu seinem Untergang bei, denn währendessen untergrub der Volkstribun Marcus Livius Drusus seine Stellung beim Volk.
Er verbreitete die Lüge, dass in Italien genug Land wäre um zwölf Kolonien zu gründen und man sich nur selbst Probleme machen würde, wenn man wie Gaius auf außeritalischem Boden Kolonisation betrieb.
Außerdem machte er das Volk darauf aufmerksam, dass man sich -- würde man Gaius' Strategie verfolgen -- die materiellen Vorteile der Römer mit den Italikern teilen müsste.
Drusus war mit seiner Taktik sehr erfolgreich, denn als Gaius wieder zurückkam, hatte er das Volk so sehr überzeugt, dass es das Gesetz zur Italikerfrage ablehnte und Gaius nicht mehr erneut zum Volkstribun wählte.\footnote{vgl. Heuss, Theodor, Römische Geschichte, S.~154}
Um Gaius völlig in den Ruin zu treiben, stellte Minucius den Antrag, die Gründung der Kolonie \textit{Junonia} rückgängig zu machen, aufgrund ungünstiger Vorzeichen.\footnote{angeblich sollten Wölfe die Grenzsteine herausgerissen haben; vgl. Bengston, Hermann, Römische Geschichte -- Republik und Kaiserzeit bis 284~n.Chr, S.~133}
Gaius Gracchus griff dann zusammen mit Fulvius Flaccus und seinen Anhängern auf nackte Gewalt zurück.
Als der Senat den \textit{senatus consultum ultimum}, also den Staatsnotstand, ausrief, verschanzten sie sich auf dem Aventin.
Zunächst entkam Gaius dem Konsul Opimius noch, lies sich dann jedoch am Tiberufer von einem Sklaven den Todesstoß geben, um nicht in die Hände der Gegner zu fallen.
Selbst die Ritter hatten sich dem Senat angeschlossen und kämpften nicht auf der Seite von Gaius, obwohl er immer großen Aufwand betrieben hatte, den Ritterstand zu seinem Verbündetem zu machen.

    \section{Nachwirkungen der Gracchen}
Neben den Brüdern Gracchus selbst, sollen noch weitere drei- bis viertausend ihrer Anhänger verfolgt und getötet worden sein.
Der Senat versuchte so schnell wie möglich, die Agrargesetze der Gracchen rückgängig zu machen.
Zunächst hob man das Verbot auf, Acker, den man durch die Ackerkommission zugeteilt bekommen hatte, zu verkaufen.
Später ging man sogar dazu über den \textit{ager occupatorius} komplett in reguläres Eigentum, ohne staatliche Rückforderungsansprüche, zu überführen. 
Insgesamt wurde zwar einigen Bauern Ackerflächen in der Zeit der Gracchen zugewiesen, aber das eigentliche Agrarproblem blieb nach ihrem Tod bestehen und wurde auch später nicht grundlegend gelöst.

Nur die Gesetze, die für den Ritterstand gemacht worden waren, lies man unangetastet.
Die Ritter waren auch diejenigen, die als klare Sieger aus der Zeit der Gracchen herauskamen, denn sie hatten nun einen eigenen Stand, der auf politischer Ebene nicht mehr unberücksichtigt bleiben konnte.\footnote{vgl. Heuss, Alfred, Römische Geschichte, S.~156}

Des weiteren resultierten aus der Zeit der Gracchen zwei grundlegende Ansätze der römischen Politik, die künftig betrieben wurde.
Zum Einen gab es die "`Popularen"', die -- mit den Gracchen als Vorbildern -- Politik mit Hilfe der Volksversammlung machten und als ihre Gegenspieler die "`Optimaten"'\footnote{"`die Besten"'}, die ihre konservative Politik von Senat und Nobilität aus betrieben.



\chapter{Retter des Staates oder Anarchisten?}
    \section{Retter des Staates}
%Betrachtet man die Zeit nach den Gracchen, so stößt man auf die Bürgerkriege, den Zerfall der \textit{res publica} und den Beginn der Kaiserzeit.
%Nach den Ergebnissen ihrer Politik zu urteilen, waren die Brüder Gracchus also definitiv keine "`Retter des Staates"'.

%Untersucht man jedoch, wie die Zustände in Rom ausgesehen hätten, wenn die Gracchen ihre Reformen erfolgreich durchgeführt hätten, so braucht es schon einiges mehr an Überlegung, um zu beantworten, ob die Gracchen wie "`Retter des Staates"' gehandelt haben.

Ich denke, die Gracchen wollten die Republik erhalten, indem sie versuchten ihre wichtigste Säule -- das Bauerntum -- zu stärken.
Hätten die Brüder Gracchus ihre Reformen durchsetzen können, so wäre dadurch vermutlich auch das Agrarproblem mit seinen Konsequenzen für das Militärwesen Roms gelöst worden.
Durch die Stärkung des Kleinbauerntums wäre die Republik in ihren alten Verhältnissen erhalten geblieben.
Meiner Meinung nach verraten diese Bestrebungen durchaus ein Interesse an der Rettung des Staates, so wie er existierte.
Stattdessen läuteten sie aber die Zeit der Bürgerkriege und mit ihnen das Ende der Republik und den Beginn der Kaiserzeit ein.
Obwohl ich denke, dass sie die Stärkung und Erhaltung der Republik zum Ziel hatten, würde ich sie nicht als "`Retter des Staates"' bezeichnen, da sie eher das Gegenteil erreichten.



Um die Frage zu beantworten, ob die Brüder Gracchus Anarchisten waren, muss man zunächst den Begriff "`Anarchist"' klären.
Ein Anarchist ist eine Person, welche den politischen Zustand der Anarchie anstrebt.
Anarchie steht für die Herrschaftslosigkeit, also einen politischen Zustand, in welchem Verfassung, Recht und Gesetz keine Geltung haben und kein Mensch über einen anderen herrscht.\footnote{vgl. Lexikon-Institut der Bertelsmann Lexikon Verlag GmbH (Hrsg.), Das neue Taschenlexikon -- Band A, S.~158}

Demnach wären die Brüder Gracchus Anarchisten, wenn sie die Verfassung der \textit{res publica} und den dort geltenden Rechten und Gesetzen keine Beachtung schenken würden.
Um das herauszufinden, sollte man ein besonderes Augenmerk auf Tiberius und seine Methoden legen, da seine Gegner seinen Mord vorallem damit rechtfertigten, dass er die Verfassung missachtet hätte.
Durch die Absetzung des Volkstribuns Marcus Octavius, missachtete er tatsächlich die römische Verfassung, da diese vorsah, dass ein Volkstribun unverletzlich sei.
Auch wenn er seinen Schritt nachvollziehbar in einer Rede vor der Volksversammlung begründete\footnote{vgl. S.~\pageref{octavius}}, bin ich der Meinung, dass er nur sein eigenes Ziel verfolgte.
Er wollte um jeden Preis das Agrargesetz durchsetzen und musste dazu zunächst seinen Gegner aus dem Weg räumen.
Auch bei seinem Versuch, sich ein zweites Mal entgegen der Verfassung wählen zu lassen, verhält es sich meines Erachtens nach ähnlich.
Nur um der Rache seiner Feinde zu entgehen, also aus seinem eigenem Interesse, versuchte er die Verfassung zu umgehen und sich ein weiteres Jahr in die Unverletzlichkeit des Volkstribunen zu retten.
Es lässt sich also durchaus sagen, dass er sich anarchistischer Methoden bediente, um seine Politik durchzusetzen.
Da er aber nicht als Ziel die Abschaffung der Verfassung, sondern im Gegenteil die Erhaltung der \textit{res publica} hatte, wäre es meiner Meinung nach zu weit gegriffen, ihn als Anarchisten zu bezeichnen.

Bei Gaius verhält es sich noch eindeutiger.
Er geht zwar teilweise ähnlich wie sein älterer Bruder vor und lässt sich zum Beispiel erfolgreich zweimal hintereinander zum Volkstribun wählen, jedoch ändert er zuvor die Verfassung so ab, dass er kein Gesetz bricht.
Somit sind bei ihm, sofern man von seinem Ende absieht -- als er gewaltsam mit seinen Anhängern gegen seine Gegner vorgeht --, keine anarchistischen Methoden zu finden, da er sich an die gegebenen Regeln der römischen Republik hält.

Letztendlich denke ich, dass man die Brüder Gracchus weder als Anarchisten noch als Retter des Staates bezeichnen kann, sondern eher als Politiker, die die Rettung des Staates zum Ziel hatten und dieses mit teilweise anarchistischen Methoden durchsetzen wollten, am Ende jedoch an ihren Gegnern scheiterten.




\begin{thebibliography}{}
 \bibitem{atlas}Stier, Hans-Erich et al. (Hrsgg.), \textit{Großer Atlas zur Weltgeschichte}. Orbis Verlag, München~1990
 \bibitem{baker}Baker, Simon, \textit{ROM -- AUFSTIEG UND UNTERGANG EINER WELTMACHT}. Philipp Reclam jun. GmbH \& Co., Stuttgart~2007
 \bibitem{bengtson}Bengtson, Hermann, \textit{RÖMISCHE GESCHICHTE -- Republik und Kaiserzeit bis 284~n.Chr.}. C.H.Beck, München~1988
 \bibitem{bleicken}Bleicken, Jochen, \textit{Geschichte der römischen Republik}. R.~Oldenbourg~Verlag, München~2004
 \bibitem{bringmann}Bringmann, Klaus (Hrsg.), \textit{Krise und Ende der römischen Republik (133-42~v.Chr)}. Akademie Verlag, Berlin~2003
 \bibitem{cancik}Cancik, Hubert et al. (Hrsgg.), \textit{Der neue Panly: Enzyklopädie der Antike}. J.B.~Metzler, Weimar~2001
 \bibitem{grimal}Grimal, Pierre (Hrsg.), \textit{Der Aufbau des römischen Reiches -- Die Mittelmeerwelt im Altertum \RM{3}}. Fischer Taschenbuch Verlag, Frankfurt am Main~1996
 \bibitem{heuss}Heuss, Alfred, \textit{RÖMISCHE GESCHICHTE}. Wissenschaftliche Buchgesellschaft, Braunschweig 5.~Auflage~1983
 \bibitem{jehne}Jehne, Martin, \textit{DIE RÖMISCHE REPUBLIK -- Von der Gründung bis Caesar}. C.H.Beck, München~2006
 \bibitem{lexikon}Lexikon-Institut der Bertelsmann Lexikon Verlag GmbH (Hrsg.), \textit{DAS NEUE TASCHEN LEXIKON}. Bertelsmann Lexikon Verlag, Gütersloh~1992
 \bibitem{vollkommer}Vollkommer, Rainer, \textit{Das römische Weltreich}. Konrad Theiss Verlag GmbH, Stuttgart~2008
 \bibitem{zimmermann}Zimmermann, Klaus, \textit{Rom und Karthago}. Wissenschaftliche Buchgesellschaft, Darmstadt~2005
 
\end{thebibliography}

\chapter*{Eidesstattliche Erklärung}

Hiermit versichere ich, dass ich die Facharbeit selbständig verfasst,
keine anderen als die angegebenen Quellen und Hilfsmittel benutzt
habe und alle Ausführungen, die aus anderen Schriften übernommen wurden,
kenntlich gemacht sind.

\vspace{2.5cm}


\begin{tabular}{ll}
{\footnotesize \rule{0.4\textwidth}{0.3pt},} & {\footnotesize \rule{0.2\textwidth}{0.3pt}}\tabularnewline
{\footnotesize Ort} & {\footnotesize Datum}\tabularnewline
 & \tabularnewline
{\footnotesize \rule{0.4\textwidth}{0.3pt}} & \tabularnewline
{\footnotesize Unterschrift} & \tabularnewline
\end{tabular}


\end{document}
